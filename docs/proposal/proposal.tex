\documentclass[
  11pt,
  a4paper,
]{article}

\title{Project Proposal}
\author{Martin de Spirlet}
\date{}

% Set document geometry
\usepackage[a4paper]{geometry} % Flexible and complete interface to document dimensions

\usepackage[numbers]{natbib} % Flexible bibliography support
\usepackage[
  skip = 1\baselineskip plus 2pt,
  indent,
]{parskip} % Layout with zero \parindent non-zero \parskip
\usepackage[onehalfspacing]{setspace} % Set space between lines % setspace must be loaded before hyperref

\PassOptionsToPackage{hyphens}{url}\usepackage{hyperref} % Extensive support for hypertext in LaTeX

\usepackage{usebib} % A simple bibliography processor % usebib must be loaded after hyperref

% Set the bibliography style (using natbib)
\bibliographystyle{IEEEtranN}

% Use bibliography fields in the document (using usebib)
\bibinput{bibliography}

\begin{document}

\maketitle

As the vast quantities of data generated by modern applications continue to increase, so too does the difficulty in storing and processing these data.
Streaming algorithms offer a way to process and query long sequences of data, typically by constructing short summaries or `sketches' of the data in a single pass.
As a result, such algorithms can only support certain queries, depending on which properties of the data they capture in their summaries.
Additionally, since some information is lost when a summary is constructed, streaming algorithms often compute approximations rather than exact answers.
Nevertheless, the use of summaries provides a significant reduction in time and space complexity~\citep{cormode20}.

The aim of this project is to investigate streaming algorithms and their applications.
In particular, the fingerprint summary, count sketch and dyadic count sketch described in~\cite{cormode20} will be implemented from scratch to create a small library in Java.
These algorithms each support one or more queries on properties related to the frequency of items in multiset data, and are chosen due to their utility and lack of implementations conveniently available online.
This is particularly true of the fingerprint summary and the dyadic count sketch.

For this project, each algorithm is to be understood theoretically, expressed formally in pseudocode, and analysed in order to reason about its correctness.
Additionally, the algorithms will be implemented in Java, and their performance and accuracy will be evaluated on both artificial and published datasets, such as those provided by~\cite{dua18} and~\cite{demaine14}.
This will be achieved using a stream processing framework, such as \usebibentry{tasf14}{title}~\citep{tasf14}.
The results will be analysed with respect to relevant literature and properties of the datasets used.

It is important that the main bulk of development is complete by the end of week~8, so that the preparation of results and their analyses can begin in time for the demonstration in week~12.
It is likely that the literature review and implementation of algorithms will proceed concurrently until the end of week~6.

\bibliography{bibliography}

\end{document}
