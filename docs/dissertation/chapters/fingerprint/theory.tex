\section{Theory}
\label{sec:fingerprint-theory}

\begin{algorithm}
  \caption{The fingerprint summary initialization operation}
  \label{alg:fingerprint-theory-initialize}
  \begin{algorithmic}[1]
  \Out the initial array \( C = \dataarray{0}_{m \times n} \) of counters
  \Constant the number \( m \in \positiveintegers \) of rows in the array; the number \( n \in \positiveintegers \) of columns in the array; the prime number \( p \in \dataset{\phi \in \primes \suchthat \phi > n} \)
  \Local the hash functions \( h_{i} \colon \integers \to \dataintegerinterval{n} \) that each map an item to a position in row~\( i \) of the array; the discriminating hash functions \( g_{i} \colon \integers \to \dataset{-1, +1} \) that each map an item to the sign adjustment of its update in row~\( i \) of the array
  \Function{Initialize}{}
    \State \( C \gets \dataarray{0}_{m \times n} \)
    \ForAll{\( i \in \dataintegerinterval{m} \)}
      \State \( \datasequence{\alpha_{i}, \beta_{i}} \gets \) pick uniformly at random from \( \dataintegerinterval{1, p - 1} \)
      \State \( \hashFunction (C, i) \gets h_{i} \colon x \mapsto ((\alpha_{i} \cdot x + \beta_{i}) \Mod p) \Mod n \)
      \State \( \datasequence{\gamma_{i}, \theta_{i}} \gets \) pick uniformly at random from \( \dataintegerinterval{1, p - 1} \)
      \State \( \discriminatorFunction (C, i) \gets g_{i} \colon x \mapsto 2 \cdot (((\gamma_{i} \cdot x + \theta_{i}) \Mod p) \Mod 2) - 1 \)
    \EndFor
    \State \Return \( C \)
  \EndFunction
\end{algorithmic}

\end{algorithm}

The fingerprint initialization operation is presented in \cref{alg:fingerprint-theory-initialize}.
A fingerprint \( f \) is initialized by setting its value to zero, and a base \( \alpha \) is drawn uniformly at random from the integer interval \( \dataintegerinterval{1, p - 1} \).
This is assigned to the primitive routine \( \base (f) \).
It is assumed that the prime number \( p > u - 1 \) has been selected beforehand.

\begin{algorithm}
  \caption{The fingerprint summary update operation}
  \label{alg:fingerprint-theory-update}
  \begin{algorithmic}[1]
  \In the fingerprint \( f \in \dataintegerinterval{0, p - 1} \) to update; the item \( x \in U \) for which to update the fingerprint; the weight \( w \in \integers \) of the update to the item
  \Out the updated fingerprint \( f' \in \dataintegerinterval{0, p - 1} \)
  \Constant the prime number \( p \in \dataset{\phi \in \primes \suchthat \phi > u} \)
  \Function{Update}{\relaxedmath{f}, \relaxedmath{x}, \relaxedmath{w}}
    \State \( f' \gets \Copy f \)
    \State \( \base (f') \gets \base (f) \)
    \State \( f' \gets (f' + w \cdot \base (f')^{x}) \Mod p \)
    \State \Return \( f' \)
  \EndFunction
\end{algorithmic}

\end{algorithm}

The fingerprint update operation is presented in \cref{alg:fingerprint-theory-update}.
This operation updates a fingerprint \( f \) so as to include in its representation of the underlying multiset an increase in the multiplicity of an item \( x \) by weight \( w \).
The value of the updated fingerprint \( f' \) is given by \cref{eq:fingerprint-theory-update}.

\begin{equation}
  \label{eq:fingerprint-theory-update}
  f' = \left( f + w \cdot \alpha^{x} \right) \bmod p.
\end{equation}

\begin{algorithm}
  \caption{The fingerprint summary merge operation}
  \label{alg:fingerprint-theory-merge}
  \begin{algorithmic}[1]
  \In the two fingerprints \( a, b \in \dataintegerinterval{0, p - 1} \) to merge (\( \base (a) = \base (b) \))
  \Out the merged fingerprint \( f \equiv a + b \pmod{p} \in \dataintegerinterval{0, p - 1} \)
  \Constant the prime number \( p \in \dataset{\phi \in \primes \suchthat \phi > u} \)
  \Function{Merge}{\relaxedmath{a}, \relaxedmath{b}}
    \State \( f \gets (a + b) \Mod p \)
    \State \( \base (f) \gets \base (a) \)
    \State \Return \( f \)
  \EndFunction
\end{algorithmic}

\end{algorithm}

Two fingerprints can be merged into one.
This operation allows the fingerprint of a stream to be computed in parallel, i.e.\@ fingerprints of distinct portions of a stream can be computed concurrently and eventually combined in order to obtain the fingerprint of the stream as a whole.
The fingerprint merge operation is presented in \cref{alg:fingerprint-theory-merge}.
Merging two fingerprints \( a \) and \( b \) into a single fingerprint \( f \) is a simple case of addition modulo \( p \), as formalized in \cref{eq:fingerprint-theory-merge}, but this will only work if the two fingerprints are computed using the same base.
This should come as no surprise, since it is a generalization of the update operation; an update is the special case of the merger of one fingerprint \( a = f \), whose base is \( \alpha \), and another fingerprint \( b = w \cdot \alpha^{x} \), whose base is also \( \alpha \)---the only difference being that in the case of an update, the fingerprint \( b \) represents a multiset whose underlying set has a cardinality of one.

\begin{equation}
  \label{eq:fingerprint-theory-merge}
  f = \left( a + b \right) \bmod p.
\end{equation}

\begin{algorithm}
  \caption{The fingerprint summary query operation}
  \label{alg:fingerprint-theory-query}
  \begin{algorithmic}[1]
  \In the array \( C \in \realnumbers^{m \times n} \) to query; the item \( x \in U \) whose approximate frequency in the multiset \( S \) to return
  \Out the approximate frequency \( f \in \realnumbers \) of the item in the multiset (\( f \simeq \cardinality{\datamultiset{s \in S \suchthat s = x}} \))
  \Constant the number \( m \in \positiveintegers \) of rows in the array
  \Local the multiset \( F \in \realnumbers^{m} \) of approximate frequencies of the item
  \Function{Query}{\relaxedmath{C}, \relaxedmath{x}}
    \State \( F \gets \varnothing \)
    \ForAll{\( i \in \dataintegerinterval{m} \)}
      \State \( j \gets (\hashFunction (C, i))(x) \)
      \State \( k \gets (\discriminatorFunction (C, i))(x) \)
      \State \( F \gets F \cup \dataset{k \cdot C_{i, j}} \)
    \EndFor
    \State \( f \gets \median F \)
    \State \Return \( f \)
  \EndFunction
\end{algorithmic}

\end{algorithm}

For the sake of completeness, the fingerprint query operation is presented in \cref{alg:fingerprint-theory-query}.
This operation simply returns the value of the fingerprint.
To approximate whether two fingerprints represent the same multiset, their values can be compared for equality, but this will only work if the two fingerprints are computed using the same base.
If the fingerprints differ, the multisets they represent must also differ.
Additionally, if the multisets are equal, their fingerprints must also be equal.
This is formalized in \cref{eq:fingerprint-theory-unequal-implication,eq:fingerprint-theory-equal-implication}.
Note that if the fingerprints are equal, it is not necessarily the case that the multisets are equal, although for sufficiently large values of \( p \), it is highly likely that they are (see \cref{subsec:fingerprint-analysis-accuracy}).

\begin{align}
  \label{eq:fingerprint-theory-unequal-implication}
  a \neq b \implies S_{a} \neq S_{b}. \\
  \label{eq:fingerprint-theory-equal-implication}
  a = b \impliedby S_{a} = S_{b}.
\end{align}

\begin{algorithm}
  \caption{The fingerprint summary equality operation}
  \label{alg:fingerprint-theory-equal}
  \begin{algorithmic}[1]
  \In the two fingerprints \( a, b \in \dataintegerinterval{0, p - 1} \) to compare (\( \base (a) = \base (b) \))
  \Out whether the fingerprints appear to represent the same multiset \( S \) (\( a \neq b \implies S_{a} \neq S_{b} \) \LAnd{} \( a = b \impliedby S_{a} = S_{b} \))
  \Function{Equal}{\relaxedmath{a}, \relaxedmath{b}}
    \State \Return \( a = b \)
  \EndFunction
\end{algorithmic}

\end{algorithm}

A fingerprint equality operation can be defined for this comparison, as presented in \cref{alg:fingerprint-theory-equal}.
This operation simply returns \True{} if the two fingerprints are equal, and \False{} otherwise.
It is assumed that the fingerprints share the same base.
