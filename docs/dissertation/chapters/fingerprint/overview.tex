\section{Overview}
\label{sec:fingerprint-overview}

A fingerprint summary is a type of identification summary; it represents the underlying multiset---which may be very large---of a stream \( \mathcal{S} \) as a single, much smaller hash value.
This allows two fingerprints to be compared for equality in order to determine whether they represent the same multiset~\citep{breslauer11}.
In this context, each datum in the stream is an ordered pair of a non-negative integer item \( x \), and a (positive or negative) integer weight \( w \).
The underlying multiset \( S \) comprises all items \( x \) that appear in the stream, and each weight \( w \) is a summand of the multiplicity of its corresponding item in the multiset.
The universe \( U \) from which items are drawn is the integer interval \( U = \dataintegerinterval{u} = \dataintegerinterval{0, u - 1} \), where \( u - 1 \) is the greatest possible value that may appear in the stream~\citep{cormode20}.
This is formalized in \cref{eq:fingerprint-overview-multiset,eq:fingerprint-overview-multiplicity}.

\begin{gather}
  \label{eq:fingerprint-overview-multiset}
  S = \datamultiset[\big]{x \in \dataintegerinterval{u} \suchthat x \in \datasequence{x, w}} \quad \forall\ \datasequence{x, w} \in \mathcal{S}. \\
  \label{eq:fingerprint-overview-multiplicity}
  \cardinality[\big]{\datamultiset{s \in S \suchthat s = x}} = \smashoperator{\sum_{\datasequence{s, w} \in \mathcal{S}}} \indicator_{\dataset{x}} (s) \cdot w \quad \forall\ x \in S.
\end{gather}

Since a multiset maps items (or indices) to multiplicities, a multiset can be represented as a vector of multiplicities.
This means that the fingerprint summary is a solution to the problem of testing for the equality of vectors or sets.
This is a common task in computer science~\citep{breslauer11}.
For example, simple database queries are frequently used to search relations for tuples with a subset of attributes that exactly match a set of given values.
Another example of this problem in database management is the computation of joins, for which, again, tuples are found that match exactly according to a subset of their attributes~\citep{karp87}.
The hash value maintained by a fingerprint summary could also be used as a checksum for validating the correct transmission of packets of data~\citep{tridgell96}.
Although the fingerprint summary is used primarily to check for equality between multisets, multiple fingerprints could be used to check for similarity, since two sets with a large number of equal subsets are likely to be similar.
The ability to check for similarity between vectors of data is useful for search engines, natural language processing and the computation of diffs between files~\citep{tridgell96}.
Although the fingerprint summary is designed to represent multisets of integers, it can be adapted to handle any arbitrary type of data, including strings, as long as a suitable hash function can be found to map each distinct datum to a unique non-negative integer hash value (see \cref{subsec:fingerprint-analysis-universe}).

A fingerprint \( f \) represents a multiset as a hash value in the integer interval \( \dataintegerinterval{0, p - 1} \), where \( p \) is a prime number greater than the upper bound \( u - 1 \) of the universe.
The hash value is computed using a rolling hash function.
This means that the hash value is easily updated for a new input, given only the old hash value and the new input~\citep{karp87}.
This is similar to the computation of a moving average.
The computation of the hash value requires a base \( \alpha \) that must be drawn uniformly at random from the integer interval \( \dataintegerinterval{1, p - 1} \).
The base can be interpreted as a property of the fingerprint.
Thus, in the following algorithms, the base of a fingerprint \( f \) is represented by the primitive routine \( \base (f) \).
Each of the following algorithms is either adapted from, or designed according to the description of, the corresponding operation in the definitive literature~\citep{cormode20}.
