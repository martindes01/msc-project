\section{Theory}
\label{sec:count-sketch-theory}

\begin{algorithm}
  \caption{The count sketch initialization operation}
  \label{alg:count-sketch-theory-initialize}
  \begin{algorithmic}[1]
  \Out the initial array \( C = \dataarray{0}_{m \times n} \) of counters
  \Constant the number \( m \in \positiveintegers \) of rows in the array; the number \( n \in \positiveintegers \) of columns in the array; the prime number \( p \in \dataset{\phi \in \primes \suchthat \phi > n} \)
  \Local the hash functions \( h_{i} \colon \integers \to \dataintegerinterval{n} \) that each map an item to a position in row~\( i \) of the array; the discriminating hash functions \( g_{i} \colon \integers \to \dataset{-1, +1} \) that each map an item to the sign adjustment of its update in row~\( i \) of the array
  \Function{Initialize}{}
    \State \( C \gets \dataarray{0}_{m \times n} \)
    \ForAll{\( i \in \dataintegerinterval{m} \)}
      \State \( \datasequence{\alpha_{i}, \beta_{i}} \gets \) pick uniformly at random from \( \dataintegerinterval{1, p - 1} \)
      \State \( \hashFunction (C, i) \gets h_{i} \colon x \mapsto ((\alpha_{i} \cdot x + \beta_{i}) \Mod p) \Mod n \)
      \State \( \datasequence{\gamma_{i}, \theta_{i}} \gets \) pick uniformly at random from \( \dataintegerinterval{1, p - 1} \)
      \State \( \discriminatorFunction (C, i) \gets g_{i} \colon x \mapsto 2 \cdot (((\gamma_{i} \cdot x + \theta_{i}) \Mod p) \Mod 2) - 1 \)
    \EndFor
    \State \Return \( C \)
  \EndFunction
\end{algorithmic}

\end{algorithm}

The count sketch initialization operation is presented in \cref{alg:count-sketch-theory-initialize}.
A count sketch is initialized by creating an \( m \times n \) array of counters and setting every element in the array to zero.
For each row \( i \) of the array, two parameters \( \alpha_{i} \) and \( \beta_{i} \) are drawn uniformly at random from the integer interval \( \dataintegerinterval{1, p - 1} \) and used to define a linear hash function \( h_{i} \colon \integers \to \dataintegerinterval{n} \), which is assigned to the primitive routine \( \hashFunction (C, i) \).
Additionally, for each row \( i \), another two parameters \( \gamma_{i} \) and \( \theta_{i} \) are drawn uniformly at random from the integer interval \( \dataintegerinterval{1, p - 1} \) and used to define a discriminating linear hash function \( g_{i} \colon \integers \to \dataset{-1, +1} \), which is assigned to the primitive routine \( \discriminatorFunction (C, i) \).
It is assumed that the number of rows \( m \), the number of columns \( n \) and the prime number \( p > n \) have been selected beforehand.

\begin{algorithm}
  \caption{The count sketch update operation}
  \label{alg:count-sketch-theory-update}
  \begin{algorithmic}[1]
  \In the fingerprint \( f \in \dataintegerinterval{0, p - 1} \) to update; the item \( x \in U \) for which to update the fingerprint; the weight \( w \in \integers \) of the update to the item
  \Out the updated fingerprint \( f' \in \dataintegerinterval{0, p - 1} \)
  \Constant the prime number \( p \in \dataset{\phi \in \primes \suchthat \phi > u} \)
  \Function{Update}{\relaxedmath{f}, \relaxedmath{x}, \relaxedmath{w}}
    \State \( f' \gets \Copy f \)
    \State \( \base (f') \gets \base (f) \)
    \State \( f' \gets (f' + w \cdot \base (f')^{x}) \Mod p \)
    \State \Return \( f' \)
  \EndFunction
\end{algorithmic}

\end{algorithm}

The count sketch update operation is presented in \cref{alg:count-sketch-theory-update}.
This operation updates a count sketch \( C \) so as to include in its summary of the underlying multiset an increase in the multiplicity of item \( x \) by weight \( w \).
For each row \( i \) of the array, the counter at position~\( h_{i} (x) \) of that row is updated by adding the product of the discriminating value \( g_{i} (x) \) and the weight \( w \).
If \( g_{i} (x) = -1 \), this is equivalent to subtracting \( w \) from the corresponding counter.
If \( g_{i} (x) = +1 \), this is equivalent to adding \( w \) to the corresponding counter.
The values of the updated count sketch \( C' \) are given by \cref{eq:count-sketch-theory-update}.

\begin{equation}
  \label{eq:count-sketch-theory-update}
  C'_{i, h_{i} (x)} = C_{i, h_{i} (x)} + g_{i} (x) \cdot w \quad \forall\ i \in \dataintegerinterval{m}.
\end{equation}

\begin{algorithm}
  \caption{The count sketch merge operation}
  \label{alg:count-sketch-theory-merge}
  \begin{algorithmic}[1]
  \In the two fingerprints \( a, b \in \dataintegerinterval{0, p - 1} \) to merge (\( \base (a) = \base (b) \))
  \Out the merged fingerprint \( f \equiv a + b \pmod{p} \in \dataintegerinterval{0, p - 1} \)
  \Constant the prime number \( p \in \dataset{\phi \in \primes \suchthat \phi > u} \)
  \Function{Merge}{\relaxedmath{a}, \relaxedmath{b}}
    \State \( f \gets (a + b) \Mod p \)
    \State \( \base (f) \gets \base (a) \)
    \State \Return \( f \)
  \EndFunction
\end{algorithmic}

\end{algorithm}

Two count sketches can be merged into one.
This operation allows the count sketch of a stream to be computed in parallel, i.e.\@ count sketches of distinct portions of a stream can be computed concurrently and eventually combined in order to obtain the count sketch of the stream as a whole.
The count sketch merge operation is presented in \cref{alg:count-sketch-theory-merge}.
Merging two count sketches \( A \) and \( B \) into a single count sketch \( C \) is a simple case of element-wise addition, as formalized in \cref{eq:count-sketch-theory-merge}, but this will only work if the two count sketches have the same size and are computed using the same sequences of hash functions.
Since they are computed using the same hash functions, items are mapped to the same positions in both count sketches.
Additionally, the same discriminating values are applied to update weights for items in corresponding positions.
Thus, the merged count sketch \( C \) is equivalent to the count sketch that would be obtained if all the updates to \( A \) and \( B \) were applied to a single count sketch.

\begin{equation}
  \label{eq:count-sketch-theory-merge}
  C_{i, j} = A_{i, j} + B_{i, j} \quad \forall\ \datasequence{i, j} \in \dataintegerinterval{m} \times \dataintegerinterval{n}.
\end{equation}

\begin{algorithm}
  \caption{The count sketch query operation}
  \label{alg:count-sketch-theory-query}
  \begin{algorithmic}[1]
  \In the array \( C \in \realnumbers^{m \times n} \) to query; the item \( x \in U \) whose approximate frequency in the multiset \( S \) to return
  \Out the approximate frequency \( f \in \realnumbers \) of the item in the multiset (\( f \simeq \cardinality{\datamultiset{s \in S \suchthat s = x}} \))
  \Constant the number \( m \in \positiveintegers \) of rows in the array
  \Local the multiset \( F \in \realnumbers^{m} \) of approximate frequencies of the item
  \Function{Query}{\relaxedmath{C}, \relaxedmath{x}}
    \State \( F \gets \varnothing \)
    \ForAll{\( i \in \dataintegerinterval{m} \)}
      \State \( j \gets (\hashFunction (C, i))(x) \)
      \State \( k \gets (\discriminatorFunction (C, i))(x) \)
      \State \( F \gets F \cup \dataset{k \cdot C_{i, j}} \)
    \EndFor
    \State \( f \gets \median F \)
    \State \Return \( f \)
  \EndFunction
\end{algorithmic}

\end{algorithm}

The count sketch query operation is presented in \cref{alg:count-sketch-theory-query}.
This operation returns an approximation of the frequency of an item \( x \) in the multiset summarized by the count sketch.
The values of the counters corresponding to \( x \) are collected from all the positions to which \( x \) is mapped, into a multiset \( F \).
The final approximation \( f \) of the frequency of \( x \) is the median value in \( F \).
This is formalized in \cref{eq:count-sketch-theory-query}.

\begin{equation}
  \label{eq:count-sketch-theory-query}
  f = \median_{i \in \dataintegerinterval{m}} \bigl( g_{i} (x) \cdot C_{i, h_{i} (x)} \bigr) \simeq \cardinality[\big]{\datamultiset{s \in S \suchthat s = x}}.
\end{equation}
