\documentclass{beamer}

\useoutertheme{infolines}

\setbeamertemplate{headline}[default]
\setbeamertemplate{footline}[infolines]
\setbeamertemplate{navigation symbols}{}

\title[Streaming Algorithms]{Theory and Proof-of-Concept Implementations \\ of Streaming Algorithms for Multiset Data}
\subtitle{MSc Computer Science Conversion Project}
\author{Martin de Spirlet}
\institute[]{University of Birmingham}
\date{Project Demonstration}

\usepackage{booktabs} % Publication quality tables in LaTeX
\usepackage{listings} % Typeset source code listings using LaTeX
\usepackage{siunitx} % A comprehensive (SI) units package

\lstset{
  basicstyle = \ttfamily,
}

\begin{document}

\begin{frame}
  \titlepage
\end{frame}

\begin{frame}
  \frametitle{Types of Test}

  \begin{block}<1->{Accuracy Test}
    \begin{itemize}
      \item Generate an artificial random dataset
      \item Construct a summary of the dataset
      \item Query the summary for approximations
      \item Determine the experimental accuracy of the approximations
      \item Compare the experimental accuracy to the theoretical accuracy
    \end{itemize}
  \end{block}

  \begin{block}<2->{Performance Test}
    \begin{itemize}
      \item Generate artificial random datasets of a range of sizes
      \item Time the construction of their summaries
      \item Determine the experimental time complexity
      \item Compare the experimental time complexity to the theoretical time complexity
    \end{itemize}
  \end{block}
\end{frame}

\begin{frame}
  \frametitle{Running the Tests}

  \begin{block}{Custom Gradle Task}
    \begin{center}
      \lstinline{./gradlew <test-name> [--project-prop options="<options>"]}
    \end{center}

    \begin{table}
      \centering
      \begin{tabular}{
        l
        S[
          table-alignment = right,
          table-format = 5,
        ]
      }
        \toprule
        Option name & {Default value} \\
        \midrule
        \lstinline{--size} & 65536 \\
        \lstinline{--item-lower-bound} & \\
        \lstinline{--item-upper-bound} & \\
        \lstinline{--weight-lower-bound} & \\
        \lstinline{--weight-upper-bound} & \\
        \lstinline{--rows} & \\
        \lstinline{--columns} & \\
        \lstinline{--queries} & 65536 \\
        \lstinline{--runs} & 1 \\
        \bottomrule
      \end{tabular}
    \end{table}
  \end{block}
\end{frame}

\begin{frame}
  \frametitle{The Fingerprint Accuracy Tests}

  \begin{block}{Parameters}
    \begin{itemize}
      \item Set item lower bound to zero
      \item Vary item upper bound
      \item Set dataset size to \num[parse-numbers=false]{2^{16}}
      \item Run each test \num{25}~times
    \end{itemize}

    \begin{table}
      \centering
      \begin{tabular}{
        S[
          table-alignment = right,
          table-format = 8,
        ]
        S[
          round-mode = figures,
          round-precision = 3,
          table-alignment = right,
          table-format = 1.2e-1,
        ]
      }
        \toprule
        {Item upper bound} & {Failure bound} \\
        \midrule
        255 & 8.363515273e-8 \\
        4095 & 1.348040611e-6 \\
        65535 & 2.157852795e-5 \\
        1048575 & 3.452663254e-4 \\
        16777215 & 5.524271084e-3 \\
        \bottomrule
      \end{tabular}
    \end{table}
  \end{block}
\end{frame}

\begin{frame}
  \frametitle{The Count Sketch Accuracy Tests}

  \begin{block}{Parameters}
    \begin{itemize}
      \item Set number of columns to \num{10}
      \item Vary number of rows
      \item Set dataset size to \num[parse-numbers=false]{2^{16}}
      \item Set number of queries to \num[parse-numbers=false]{2^{16}}
      \item Run each test \num{25}~times
    \end{itemize}

    \begin{table}
      \centering
      \begin{tabular}{
        S[
          table-alignment = right,
          table-format = 2,
        ]
        S[
          round-mode = figures,
          round-precision = 3,
          table-alignment = right,
          table-format = 1.2e-1,
        ]
      }
        \toprule
        {Rows} & {Failure bound} \\
        \midrule
        3 & 0.04978706837 \\
        5 & 6.737946999e-3 \\
        7 & 9.118819656e-4 \\
        9 & 1.234098041e-4 \\
        11 & 1.670170079e-5 \\
        \bottomrule
      \end{tabular}
    \end{table}
  \end{block}
\end{frame}

\begin{frame}
  \frametitle{The Dyadic Count Sketch Accuracy Tests}

  \begin{block}{Parameters}
    \begin{itemize}
      \item Set count sketch size to \( 7 \times 60 \)
      \item Vary weight upper bound
      \item Error should be at most \SI{25}{\percent} of cardinality
      \item Set dataset size to \num[parse-numbers=false]{2^{16}}
      \item Set number of queries to \num[parse-numbers=false]{2^{16}}
      \item Run each test \num{25}~times
    \end{itemize}

    \begin{table}
      \centering
      \begin{tabular}{
        S[
          table-alignment = right,
          table-format = 4,
        ]
      }
        \toprule
        {Weight upper bound} \\
        \midrule
        16 \\
        64 \\
        256 \\
        1024 \\
        4096 \\
        \bottomrule
      \end{tabular}
    \end{table}
  \end{block}
\end{frame}

\end{document}
