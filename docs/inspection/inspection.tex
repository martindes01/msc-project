\documentclass{beamer}

\useoutertheme{infolines}

\setbeamertemplate{headline}[default]
\setbeamertemplate{footline}[infolines]
\setbeamertemplate{navigation symbols}{}

\title[Streaming Algorithms]{Theory and Proof-of-Concept Implementations \\ of Streaming Algorithms for Multiset Data}
\subtitle{MSc Computer Science Project}
\author{Martin de Spirlet}
\institute[]{University of Birmingham}
\date{Project Inspection}

\begin{document}

\begin{frame}
  \titlepage
\end{frame}

\section{Background}

\begin{frame}
  \frametitle{Background}

  \begin{block}{Problem}
    \begin{itemize}
      \item Modern applications generate big data
      \item Difficult to store and process
    \end{itemize}
  \end{block}

  \begin{block}{Solution}
    \begin{itemize}
      \item Short summaries or `sketches' constructed in a single pass
      \item Trade accuracy for reduced space and time complexity
    \end{itemize}
  \end{block}

  \begin{block}{Applications}
    \begin{itemize}
      \item Medicine
      \item Science
      \item Business
      \item Social media
    \end{itemize}
  \end{block}
\end{frame}

\section{Aims and Objectives}

\begin{frame}
  \frametitle{Aims and Objectives}

  \begin{block}{Aims}
    \begin{itemize}
      \item Investigate streaming algorithms and their applications.
      \item Implement the \alert{fingerprint}, \alert{count sketch} and \alert{dyadic count sketch} from scratch to create a small library in Java.
    \end{itemize}
  \end{block}

  \begin{block}{Objectives}
    Each algorithm is to be
    \begin{itemize}
      \item understood theoretically,
      \item expressed formally in pseudocode,
      \item analysed in order to understand its correctness,
      \item implemented in Java, and
      \item tested/evaluated on standard datasets.
    \end{itemize}
  \end{block}
\end{frame}

\end{document}
